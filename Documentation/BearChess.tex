
\documentclass[11pt,a4paper]{article}
\usepackage{graphicx}
\usepackage[export]{adjustbox}
\usepackage{float}
\usepackage{titlesec}
\usepackage{fancyhdr}
\usepackage{xcolor}
\usepackage[T1]{fontenc} 
\usepackage{mathptmx}
\usepackage{nameref}
\usepackage[
colorlinks=true,
urlcolor=blue,
linkcolor=black
]{hyperref}
\pagestyle{fancy}
\setcounter{secnumdepth}{4}
\setcounter{tocdepth}{4}
\fancyfoot[C]{\thepage}
\fancyfoot[L,R]{}
\fancyhead[C]{}
\fancyhead[L]{BearChess GUI}
\fancyhead[R]{Version 0.5.5.3}
\title{Chess GUI with support for electronic chessboards}
\author{Lars Nowak}
\date{10.11.2021} 


\begin{document}
\maketitle

\begin{abstract}
\textbf{Why yet another chess GUI?}\\

Many GUIs support electronic chessboards, but do not use the full potential that chessboards with piece recognition offer. They could be much better used for training or analysis of games and positions. Read more in chapter \textbf{\ref{AnalyzeMode}  \nameref{AnalyzeMode}} on page \pageref{AnalyzeMode}.

Another feature is the extended engine support. Read more in chapter \textbf{\ref{ExtendedSupport}  \nameref{ExtendedSupport}} on page \pageref{ExtendedSupport}.

So the focus of BearChess is more on exploiting the possibilities of the chessboards than being just another GUI. Of course, you not need an electronic chessboard to use BearChess.

BearChess supports the chessboards from Certabo and the boards connected via Millennium ChessLink.\\

As you can see from the version number, this software is still under development. There are still some functions that are not fully implemented and there are certainly still many bugs. But BearChess has now reached a level where it can be used and feedback from other users is welcome.\\

I am a professional programmer and write the software in my spare time and it is free. I am not an employee of Inventhio Srl trading (Certabo) or Millennium 2000 GmbH. If something does not work, Certabo or Millennium is not responsible for it.\\

Send errors, comments, suggestions for improvement or requests\\to \textbf{lars@solanosoft.com}.

\end{abstract}

\newpage
\tableofcontents
\newpage


\section{Quick start}
\begin{itemize}
	\item Simply unpack the file BearChessWin.zip into a new folder.
	\item Start BearChess with a double-click on BearChessWin.exe
	\item Connect your electronic chessboard to your computer.
	\item Set all chessmen to their start position.
	\item Configure the electronic chessboard connection (\textbf{\ref{ElectronicChessBoard}  \nameref{ElectronicChessBoard}} on page \pageref{ElectronicChessBoard}).
	\item Connect to the electronic chessboard.
	\item Load a chess engine (\textbf{\ref{InstallEngine}  \nameref{InstallEngine}} on page \pageref{InstallEngine})
	\item Start a game and make your first move on the eletronic chessboard.
\end{itemize}


\section{Introduction}
BearChess offers among others the following functions:
\begin{enumerate}
	  \item Play with Certabo chessboards.
	  \item Play with Millennium chessboards via ChessLink modul.
  	  \item Play against human beings or UCI engines.
  	  \item Play against an UCI engine in relaxed mode. "Teddy" will intervene in some places to give you a chance even against stronger chess programs. Read more in chapter \textbf{\ref{RelaxedMode}  \nameref{RelaxedMode}} on page \pageref{RelaxedMode}.  	  
  	  \item Use Certabo Avatar UCI engines.  	  
  	  \item Use MessChess chess computer emulation by Franz Huber.  	    	  
  	  \item Analyze your games or trainings with the help of electronic chessboards.
  	  \item Use multiple chess engines at the same time for playing and analyzing.
  	  \item Engine duels and tournaments.
  	  \item Support for Polyglot and Arena opening books.
  	  \item Save and load your games.
  	  \item Individual chessmen and board fields.
\end{enumerate}

BearChess follows the design of Single Document Interface (SDI). You can place different windows, e.g. chess engine output or chess move list, anywhere on your Windows desktop. If you close them or exit BearChess, the position is saved and set to the same position when reopening.\\\\
The version numbering follows the following scheme:\\ 
"\textit{major release}"\textbf{.}""\textit{minor release}"\textbf{.}""\textit{sub release}"\textbf{.}""\textit{bug fix}"\\
Currently the \textit{major release} is still 0 and will only become a 1 when all planned functions are implemented. The \textit{minor release} is increased when new features are added. The \textit{sub release} is increased when minor features are added or improved. The last number is incremented if only errors were corrected.

\subsection{Modes}
BearChess is running in different modes. The current mode is displayed in the lower left corner.
\begin{itemize}
	
	\item \textbf{Easy playing} This is the mode in the beginning. You just can simply start making your moves on the screen or on your electronic chessboard, almost without regard to the chess rules. In addition to support you can start chess programs or load opening books. But these only give hints, but do not play as opponents. This mode is automatically set if you are not playing in another mode. It is similar to the analyze mode but let you more easily start a new game from any position.
	\begin{figure}[H]
		\centering
		\includegraphics[scale=1.0]{ModeEasyPlaying.png}
		\caption{Easy playing}
		\label{fig:ModeEasyPlaying}
	\end{figure}
	\item \textbf{Playing a game} This is the mode if you play against a chess engine or another player. Only valid chess moves are allowed and the game is time controlled.
	\begin{figure}[H]
		\centering
		\includegraphics[scale=1.0]{ModePlayingAGame.png}
		\caption{Playing a game}
		\label{fig:ModePlayingAGame}
	\end{figure}
	\item \textbf{Analyzing} If you select this mode you can make any chess moves or place the pieces as you like, almost without regard to the chess rules. This mode is recommended to analyze a game or positions. Try different variants on the board and let several chess programs analyze the positions simultaneously. 
	\begin{figure}[H]
		\centering
		\includegraphics[scale=1.0]{ModeAnalyzing.png}
		\caption{Analyzing}
		\label{fig:ModeAnalyzing}
	\end{figure}
	\item \textbf{Analyzing a game} If you have loaded a played game and connected to an electronic chessboard, you can analyze the game. Follow the played moves or make some variants and let them be analyzed by different chess programs.
    \begin{figure}[H]
	\centering
	\includegraphics[scale=1.0]{ModeAnalyzing.png}
	\caption{Analyzing}
	\label{fig:ModeAnalyzingGame}
\end{figure}
	\item \textbf{Setup Position} Build up a new starting position on the chessboard. It is easiest to set it up on the electronic chessboard.
	\begin{figure}[H]
		\centering
		\includegraphics[scale=1.0]{ModeSetupPosition.png}
		\caption{Setup Position}
		\label{fig:ModeSetupPosition}
	\end{figure}
\end{itemize}


\section{Main window}

The first start of BearChess shows the following main window:
\begin{figure}[H]
	\centering
	\includegraphics[scale=0.7]{BearChess_MainWindow.png}
	\caption{Main window}
	\label{fig:MainWindow}
\end{figure}

Two buttons are active:
\begin{itemize}
  \item \includegraphics[scale=0.5]{arrow_rotate_anticlockwise.png} rotates the board.
  \item \includegraphics[scale=0.2]{bearchess_2.png} is an easy way to play a game.
Read more in chapter \textbf{\ref{easyStart}  \nameref{easyStart}} on page \pageref{easyStart}.
\end{itemize}



\subsection{Actions}
\begin{figure}[H]
	\centering
	\includegraphics[scale=1.0]{Actions.png}
	\caption{Actions}
	\label{fig:Actions}
\end{figure}
\begin{itemize}
	\item \includegraphics[scale=0.2]{bearchess.png} \textbf{Play a new game} opens a new window to select opponents and time control (see chapter \textbf{\ref{SelectOpponent}  \nameref{SelectOpponent}} on page \pageref{SelectOpponent}).
	
	\item \includegraphics[scale=0.5]{robot.png} \textbf{Run in analyze mode} allows you to analyze games or positions with suppport of several chess engines (see chapter \textbf{\ref{AnalyzeMode}  \nameref{AnalyzeMode}} on page \pageref{AnalyzeMode})
	
	\item \includegraphics[scale=0.5]{6-2-chess-png.png} \textbf{Engine duel} starts and manage engine duels (see chapter )
	
	\item \includegraphics[scale=0.5]{cup_gold.png} \textbf{Engine tournament} starts and manage engine tournaments (see chapter )
	
	\item \includegraphics[scale=0.03]{Chess_Board.png} see chapter \textbf{\ref{SetupPosition}  \nameref{SetupPosition}} on page \pageref{SetupPosition}.
	
	\item \textbf{Run startup game on start} immediately starts a new game when you start BearChess. For more information read chapter \textbf{\ref{startupgame}  \nameref{startupgame}} on page \pageref{startupgame}.
	
	\item \includegraphics[scale=0.5]{door_out.png} \textbf{Close} exits BearChess.
\end{itemize}

\subsection{Games}
\begin{figure}[H]
	\centering
	\includegraphics[scale=1.0]{Games1.png}
	\caption{Games}
	\label{fig:Games}
\end{figure}
\begin{itemize}
	\item \includegraphics[scale=0.5]{diskette.png} \textbf{Save} your current game.
	\item \includegraphics[scale=0.5]{file_manager.png}  \textbf{Games manager} opens a new window in which you can see and load all your previously saved games.
	\item \includegraphics[scale=0.5]{clipboard_sign_out.png}  \textbf{Copy to clipboard} copy the current game (PGN) to your clipboard.	
	\item \includegraphics[scale=0.5]{clipboard_sign.png}  \textbf{Paste from clipboard} loads a game (PGN) from your clipboard.
\end{itemize}
All games are saved in a database file. Read more on chapter \textbf{\ref{games}  \nameref{games}} on page \pageref{games}.

\subsection{Engines}
\begin{figure}[H]
	\centering
	\includegraphics[scale=1.0]{Engines.png}
	\caption{Engines}
	\label{fig:Engines}
\end{figure}
\begin{itemize}
	\item  \includegraphics[scale=0.5]{robot.png} \textbf{Load \& manage engines} opens a new window in which you can install, load or configure your chess engines.
	\item \textbf{Load last engine on start} if you always want to load immediately an engine when you start BearChess. It has no effect if you have activated the option "Run startup game on start".
	\item \textbf{Show communication window} opens a new window where you can follow the communication between BearChess and chess engines. It is useful to detect any problems in communication.
	\item \textbf{Settings} to configure if you want to see more information from the engine, e.g. nodes per second.	
\end{itemize}

Read more on chapter \textbf{\ref{loadEngines}  \nameref{loadEngines}} on page \pageref{loadEngines}.

\subsection{Books}
\begin{figure}[H]
	\centering
	\includegraphics[scale=1.0]{Books.png}
	\caption{Books}
	\label{fig:Books}
\end{figure}
\begin{itemize}
	\item \includegraphics[scale=0.5]{books_stack.png} \textbf{Load \& manage opening books} opens a new window in which you can install or load  your opening books.
\end{itemize}
BearChess can handle Polyglot and Arena opening books. Read more on chapter \textbf{\ref{OpeningBooks}  \nameref{OpeningBooks}} on page \pageref{OpeningBooks}.

\subsection{Settings}
\begin{figure}[H]
	\centering
	\includegraphics[scale=1.0]{Settings.png}
	\caption{Settings}
	\label{fig:Settings}
\end{figure}
\begin{itemize}
	\item  \includegraphics[scale=0.9]{Board2DPieces32.png} \textbf{Board \& Pieces} opens a window where you can change the appearance of the chessboard and the pieces.  Read more on chapter \textbf{\ref{BoardAndPieces}  \nameref{BoardAndPieces}} on page \pageref{BoardAndPieces}.
	\item \includegraphics[scale=0.5]{arrow_rotate_anticlockwise.png}  \textbf{Rotate board} if the board on the screen should automatically rotate for your color.
	\item  \includegraphics[scale=0.5]{text_list_numbers.png} \textbf{Notation/Moves} opens a windows where you can change the appearance of the notation, e.g. figurine or letters.
	\item  \includegraphics[scale=0.5]{digit_separator.png}  \textbf{Clocks} switches between large and small clocks.
	\item  \includegraphics[scale=0.5]{balance_unbalance.png}  \textbf{Captured pieces} shows the captured pieces window at startup or on demand and you can change the font size to small or big.
	\item  \includegraphics[scale=0.5]{user_silhouette.png}  \textbf{Player} opens a dialog where you can give it a first and last name.	
    \begin{figure}[H]
	   \centering
    	\includegraphics[scale=1.0]{Player.png}
	    \caption{Player}
 	   \label{fig:Player}
	\end{figure}	
\end{itemize}

\subsection{Electronic Boards}
\begin{figure}[H]
	\centering
	\includegraphics[scale=1.0]{ElectronicBoards.png}
	\caption{Electronic Boards}
	\label{fig:ElectronicBoards}
\end{figure}
\begin{itemize}
	\item  \includegraphics[scale=0.1]{Certabo_icon.png} \textbf{Certabo} opens a window where you can configure and connect to Certabo chessboards.  Read more on chapter \textbf{\ref{ConfigureCertabo}  \nameref{ConfigureCertabo}} on page \pageref{ConfigureCertabo}.
	\item  \includegraphics[scale=0.05]{Millennium ChessLink.png} \textbf{Millennium ChessLink} opens a window in which you can configure chessboards connected to Millennium ChessLink and connect to them.  Read more on chapter \textbf{\ref{ConfigureChessLink}  \nameref{ConfigureChessLink}} on page \pageref{ConfigureChessLink}.
	\item \textbf{Connect on startup} tries to connect on to the last connected chessboard when BearChess is started. 	
	\item \textbf{Black pieces in front} assumes that you have placed the black chessmen in front of you.
	\item \textbf{Start a new game on base position} recognizes when you reset all the pieces to the base position during a game. In this case, a new game will be started automatically.
	\item \textbf{Show best move in analysis mode} displays the current best move of the engine on your electronic chessboard.
\end{itemize}

\subsection{Windows}
\begin{figure}[H]
	\centering
	\includegraphics[scale=1.0]{Windows.png}
	\caption{Windows}
	\label{fig:Windows}
\end{figure}
\begin{itemize}
	\item \textbf{Show} brings clocks or move list windows to the foreground if they are currently not visible.
	\item \textbf{Arrange} auto arange all windows to fit on your screen and not overlapping.
	\item \textbf{Captured pieces} shows the captured pieces. Either all or as a difference. 
\end{itemize}

\section{Install, Configure and Load a chess engine} \label{loadEngines}

\begin{figure}[H]
	\centering
	\includegraphics[scale=1.0]{LoadEngine.png}
	\caption{Open Load And Manage UCI Engines window}
	\label{fig:LoadEngine}
\end{figure}
BearChess does not include a chess engine. Click on "Load \& manage engines" to install and configure one. "\textit{Install}" means to make a chess program BearChess known, not to install it on your computer. BearChess supports any UCI engine.\\
\begin{figure}[H]
	\centering
	\includegraphics[scale=1.0]{LoadManageEngine1.png}
	\caption{Load And Manage UCI Engines}
	\label{fig:LoadManageEngine1}
\end{figure}

\begin{itemize}
	\item \includegraphics[scale=0.5]{robot.png} Load selected engine
	\item \includegraphics[scale=0.5]{cog.png} Configure selected engine
	\item \includegraphics[scale=0.5]{file_extension_exe.png} Install a new engine
	\item \includegraphics[scale=0.5]{bin.png} Uninstall selected engine
	\item \includegraphics[scale=0.5]{door_out.png} Close the window
\end{itemize}

\subsection{Install a new engine} \label{InstallEngine}

To install a new engine click on \includegraphics[scale=0.5]{file_extension_exe.png} and select an UCI engine file, e.g. the Wasp exe file. Or you just drag \& drop the exe file onto the button.
\begin{figure}[H]
	\centering
	\includegraphics[scale=1.0]{DropEngine.png}
	\caption{Drop an engine file}
	\label{fig:DropEngine}
\end{figure}

If the file detected as UCI engine, confirm your selection.\\
\begin{figure}[H]
	\centering
	\includegraphics[scale=1.0]{uciConfirm.png}
	\caption{Confirm UCI}
	\label{fig:uciConfirm}
\end{figure}
Next, a configuration dialog box appears where you can configure the engine and give it a name.
\begin{figure}[H]
	\centering
	\includegraphics[scale=0.9]{ConfigureWasp.png}
	\caption{Engine configuration}
	\label{fig:ConfigurWasp}
\end{figure}
The name is freely assignable, must be unique across all engines. But this way you can install the same engine several times with different configurations. The configuration values, names and possibilities are given by the engines. The first time these are the default values.
\begin{itemize}
	\item \includegraphics[scale=0.5]{accept_button.png} Accept the changes
	\item \includegraphics[scale=0.5]{file_extension_log.png} Opens the folder where the log and configration files for this engine are located
	\item \includegraphics[scale=0.5]{undo.png} Reset to default values
	\item \includegraphics[scale=0.5]{cancel.png} Cancel
\end{itemize}
The small \includegraphics[scale=0.3]{folder.png} button, e.g. 'OwnBook\_File', opens a file or directory selection dialog, depending on the configuration name (file, path, dir).\\
The configuration type 'button', e.g. 'Clear Hash', works only if the engine is loaded.

\subsubsection{Assign a logo file}
The small \includegraphics[scale=0.3]{folder.png} button on the 'Logo:' line opens a file selection dialog where you can choose a logo file for this engine.

\begin{figure}[H]
	\centering
	\includegraphics[scale=0.9]{logoFile.png}
	\caption{Engine Logo file}
	\label{fig:LogoFile}
\end{figure}

\subsubsection{Add start parameter}
You can enter a start parameter if the engine needs one.
\begin{figure}[H]
	\centering
	\includegraphics[scale=0.9]{engineParameter.png}
	\caption{Engine Paramater}
	\label{fig:EngineParameter}
\end{figure}

\subsubsection{Use opening book}
Some engines comes with there own opening book and you have an option to use them or not. You can also tell BearChess to use an opening book before the moves are calculated by the engines. You can configure how BearChess determines the book move. 
\begin{itemize}
	\item \textbf{Best} Always chooses the best move
	\item \textbf{Flexible} Selects one of the best moves
    \item \textbf{Wide} Selects any book move
\end{itemize}
Look at chapter \textbf{\ref{OpeningBooks}  \nameref{OpeningBooks}} on page \pageref{OpeningBooks} how to install an opening book.

\subsubsection{Install Certabo Avatar UCI Chess engine}
Certabo provides a special engine, called Avatar UCI Chess engine (look at \url{https://www.avataruci.org/}).\\
This engine is not intended for super strong player but tries to replicate your style. It learn from your games (you do no need many many thousand), it can start with some tens or hundreds. Certabo have an learning alghrithm which create some weights and opening books emulating personality. Certabo created avatar of some player from the past and you can find in the latest Certabo software already just select Avatar as main engine and you will be prompted to choose the Avatar you want to play.\\
You can use this engine in BearChess, too. 
If you have installed the latest Certabo software (Version 4.5 or above), select the avatar.exe file from the engines subfolder as the new engine. BearChess looks at the 'avatar\_weights' subfolder and ask you to select an avatar.

\begin{figure}[H]
	\centering
	\includegraphics[scale=0.9]{avatar1.png}
	\caption{Select an avatar}
	\label{fig:Avatar1}
\end{figure}

\begin{figure}[H]
	\centering
	\includegraphics[scale=0.9]{avatar2.png}
	\caption{Confirm}
	\label{fig:Avatar2}
\end{figure}

\begin{figure}[H]
	\centering
	\includegraphics[scale=0.7]{avatar3.png}
	\caption{Configure}
	\label{fig:Avatar3}
\end{figure}

There are no other configuration options. In the bottom line you can see the required parameters for the avatar engine.\\
If you don't find the right avatar, click on the small folder icon and you can select another avatar file, e.g. your personally created file from Certabo.

\begin{figure}[H]
	\centering
	\includegraphics[scale=0.7]{avatar4.png}
	\caption{Select your avatar}
	\label{fig:Avatar4}
\end{figure}

\subsubsection{Install MessChess engine}
Franz Huber provides emulator software to play against old chess computers as UCI engine (look at \url{https://fhub.jimdofree.com/}).\\
If you have downloaded und unzipped the latest version of CB-Emu, select the MessChess.exe file from the MessChess subfolder as the new engine. BearChess reads the Engines.lst file and ask you to select a chess computer emulation.

\begin{figure}[H]
	\centering
	\includegraphics[scale=0.9]{MessChess1.png}
	\caption{Select an emulation}
	\label{fig:MessChess1}
\end{figure}

\begin{figure}[H]
	\centering
	\includegraphics[scale=0.9]{MessChess2.png}
	\caption{Confirm}
	\label{fig:MessChess2}
\end{figure}

There are only some configuration options. In the bottom line you can see the required parameters for the engine.

\begin{figure}[H]
	\centering
	\includegraphics[scale=0.8]{MessChess3.png}
	\caption{Configure}
	\label{fig:MessChess3}
\end{figure}

\subsection{Configure an engine}

To change the configuration of an installed engine click on \includegraphics[scale=0.5]{cog.png}\\
The same configuration dialog box as during installation is shown where you can configure the engine or just change the name.

\subsection{Additional configuration for an installed engine}

If you want to save the same program with a different configuration, e.g. an additional configuration with adjusted Elo strength, you can save the configuration under a different name.

\begin{figure}[H]
	\centering
	\includegraphics[scale=0.8]{ConfigureWasp_2.png}
	\caption{Save as new configuration}
	\label{fig:LoadEngine3}
\end{figure}
Click on \includegraphics[scale=0.5]{cog_add.png} to save the configuration with a new name.\\


\subsection{Load an engine}

To load an installed engine, select an engine and click on \includegraphics[scale=0.5]{robot.png} or just double-click.

\begin{figure}[H]
	\centering
	\includegraphics[scale=1.0]{LoadEngine2.png}
	\caption{Some installed engines}
	\label{fig:LoadEngine2}
\end{figure}

\subsubsection{Load a MessChess engine}

When you load a MessChess engine the emulation window will appears, too.
\begin{figure}[H]
	\centering
	\includegraphics[scale=0.7]{MessChess4.png}
	\caption{Academy}
	\label{fig:MessChess4}
\end{figure}
Use only BearChess or your electronic chessboard to make any moves. You can use the emulaton window to set options, e.g. levels or time control.\\
{\color{red}\textbf{*}} When you play a game against a MessChess engines, the selected time control for the game cannot transferred to the emulation. To avoid that BearChess ends the game too early by timeout, you should select here only an average time per move.

\section{Load and Manage Opening Books} \label{OpeningBooks}

\begin{figure}[H]
	\centering
	\includegraphics[scale=1.0]{books.png}
	\caption{Open Load and Manage Opening Books  }
	\label{fig:LoadManageBooks1}
\end{figure}

BearChess does not include opening books. Click on "Load \& manage opening books" to install one. "\textit{Install}" means to make a opening book BearChess known, not to install it on your computer. BearChess supports Polyglot and Arena opening books.\\

\begin{figure}[H]
	\centering
	\includegraphics[scale=1.0]{LoadManageBooks2.png}
	\caption{Load and Manage Opening Books  }
	\label{fig:LoadManageBooks2}
\end{figure}

\begin{itemize}
	\item \includegraphics[scale=0.5]{book_open.png} Load selected book
	\item \includegraphics[scale=0.5]{book_add.png} Install a new book
	\item \includegraphics[scale=0.5]{bin.png} Uninstall selected book
	\item \includegraphics[scale=0.5]{door_out.png} Close the window
\end{itemize}

To install a new opening book click on \includegraphics[scale=0.5]{book_add.png} and select a book file. The file extension for Polyglot books is \textbf{bin} and for Arena is \textbf{abk}.

\begin{figure}[H]
	\centering
	\includegraphics[scale=1.0]{LoadManageBooks3.png}
	\caption{Some installed books }
	\label{fig:LoadManageBooks3}
\end{figure}

To select an opening book click on \includegraphics[scale=0.5]{book_open.png} or just double-click. 
A new window opens and shows the current possible moves found in the book.

\begin{figure}[H]
	\centering
	\includegraphics[scale=1.0]{OpeningBook.png}
	\caption{Loaded opening book on base position }
	\label{fig:OpeningBook}
\end{figure}

You can load more than one book. Every book has their own window and is synchronized with the current position on the chessboard.

\begin{figure}[H]
	\centering
	\includegraphics[scale=0.8]{OpeningBook2.png}
	\caption{Loaded opening books }
	\label{fig:OpeningBook2}
\end{figure}

{\color{red}\textbf{*}} So far, the possibilities are still very limited with the opening books. This will improve in the next versions. 

\section{Configure Board and Pieces} \label{BoardAndPieces}

\begin{figure}[H]
	\centering
	\includegraphics[scale=1.0]{SettingsBoardAndPieces.png}
	\caption{Open window to configure board and pieces }
	\label{fig:SettingsBoardAndPieces}
\end{figure}

To change the appearance of BearChess select "Board and Pieces". A new window opens where your can configure it. BearChess comes with one set of pieces and board colors.

\begin{figure}[H]
	\centering
	\includegraphics[scale=0.9]{SettingsBoardAndPieces2.png}
	\caption{Setup board and pieces }
	\label{fig:SettingsBoardAndPieces2}
\end{figure}


\begin{itemize}
	\item \includegraphics[scale=0.5]{file_manager.png} Install new board colors or pieces
	\item \includegraphics[scale=0.5]{bin.png} Uninstall board colors or pieces
	\item \includegraphics[scale=0.5]{accept_button.png} Accept the changes
	\item \includegraphics[scale=0.5]{cancel.png} Cancel
\end{itemize}

If the options "Show last move" and "Show best move" are activated, the last move and the currently best analysis move of an engine are marked on the board.

\begin{figure}[H]
	\centering
	\includegraphics[scale=0.8]{LastMove.png}
	\caption{Last move }
	\label{fig:LastMove}
\end{figure}


\begin{figure}[H]
	\centering
	\includegraphics[scale=0.8]{BestMove.png}
	\caption{Currently best move }
	\label{fig:BestMove}
\end{figure}

\subsection{Install new board colors and pieces}
BearChess uses png files for board colors and pieces.

\subsubsection{New board colors}
Click on \includegraphics[scale=0.5]{file_manager.png} to select a directory where the files are located. BearChess accepts \verb|w.png| or \verb|white.png| for white fields and \verb|b.png| or \verb|black.png| for black fields.

\begin{figure}[H]
	\centering
	\includegraphics[scale=1.0]{WoodPieces.png}
	\caption{Example for wood fields }
	\label{fig:WoodPieces}
\end{figure}

If BearChess find both files inside the directory it builds an empty chessboard to confirm your choice. A name for your board is required.

\begin{figure}[H]
	\centering
	\includegraphics[scale=1.0]{ConfirmBoard.png}
	\caption{Confirm new board }
	\label{fig:ConfirmBoard}
\end{figure}

\subsubsection{New pieces}
There a two ways to install a new set of pieces. One png file for each piece or one png file with all pieces inside. \\
Click on \includegraphics[scale=0.5]{file_manager.png} to select a directory where the files are located.\\ \textbf{Important:} The png files must have a transparent background color, otherwise they would paint over the fields.
BearChess accepts following names for the different pieces:
\begin{itemize}
	\item \textbf{White king:} \verb|KingW.png, WhiteKing.png, wk.png|
	\item \textbf{Black king:} \verb|KingB.png, BlackKing.png, bk.png|
	\item \textbf{White queen:} \verb|QueenW.png, WhiteQueen.png, wq.png|
	\item \textbf{Black queen:} \verb|QueenB.png, BlackQueen.png, bq.png|
	\item \textbf{White rook:} \verb|RookW.png, WhiteRook.png, wr.png|
	\item \textbf{Black rook:} \verb|RookB.png, BlackRook.png, br.png|
	\item \textbf{White bishop:} \verb|BishopW.png, WhiteBishop.png, wb.png|
	\item \textbf{Black bishop:} \verb|BishopB.png, BlackBishop.png, bb.png|
	\item \textbf{White knight:} \verb|KnightW.png, WhiteKnight.png, wn.png|
	\item \textbf{Black knight:} \verb|KnightB.png, BlackKnight.png, bn.png|
	\item \textbf{White pawn:} \verb|PawnW.png, WhitePawn.png, wp.png|
	\item \textbf{Black pawn:} \verb|PawnB.png, BlackPawn.png, bp.png|
\end{itemize}


\begin{figure}[H]
	\centering
	\includegraphics[scale=0.7]{Pieces.png}
	\caption{Example one png file for each piece}
	\label{fig:Pieces}
\end{figure}

If BearChess find all files inside the directory it builds an piece set to confirm your choice.\\

If BearChess find only one png file inside the directory it assumes that this file contains all pieces at once.

\begin{figure}[H]
	\centering
	\includegraphics[scale=6.5]{Leipzig.png}
	\caption{One png file with all pieces}
	\label{fig:Leipzig}
\end{figure}

The png must have the pieces in the order and colors shown above.\\
If you have one png file for all pieces, you can just drag \& drop the png file onto the
open file dialog icon. It avoids the effort of having to have a separate directory for each file.
\begin{figure}[H]
	\centering
	\includegraphics[scale=0.9]{DropPieceSet.png}
	\caption{Drop new pieces }
	\label{fig:DropPieceSet}
\end{figure}

\begin{figure}[H]
	\centering
	\includegraphics[scale=0.9]{ConfirmPieces.png}
	\caption{Confirm new pieces }
	\label{fig:ConfirmPieces}
\end{figure}

BearChess builds an piece set to confirm your choice. A name for your set is required.
Now you can combine your boards with your pieces.

\begin{figure}[H]
	\centering
	\includegraphics[scale=0.8]{CombineBoardPieces.png}
	\caption{Combine board and pieces }
	\label{fig:CombineBoardPieces}
\end{figure}


\section{Configure Notation and Moves}
\begin{figure}[H]
	\centering
	\includegraphics[scale=0.9]{NotationAndMoves1.png}
	\caption{Opens a window to configure notation and moves }
	\label{fig:NotationAndMoves}
\end{figure}
BearChess can display moves in different ways.
\begin{figure}[H]
	\centering
	\includegraphics[scale=1.0]{NotationAndMoves2.png}
	\caption{Configure notation and moves }
	\label{fig:NotationAndMoves}
\end{figure}
In long and short notation and with symbols or letters for the chessmen.

\section{Configure clock style}
\begin{figure}[H]
	\centering
	\includegraphics[scale=1.0]{ConfigureClocks.png}
	\caption{Configure clocks }
	\label{fig:ConfigureClocks}
\end{figure}

BearChess offers two different clocks: small and big

\begin{figure}[H]
	\centering
	\includegraphics[scale=1.0]{SmallClocks.png}
	\caption{Small clocks }
	\label{fig:SmallClocks}
\end{figure}
\begin{figure}[H]
	\centering
	\includegraphics[scale=0.8]{BigClocks.png}
	\caption{Big clocks }
	\label{fig:BigClocks}
\end{figure}

\section{Show captured pieces}
\begin{figure}[H]
	\centering
	\includegraphics[scale=1.0]{CapturedPieces1.png}
	\caption{Show captured pieces }
	\label{fig:CapturedPieces1}
\end{figure}

Opens a window that shows the captured pieces.
\begin{figure}[H]
	\centering
	\includegraphics[scale=1.0]{CapturedPieces2.png}
	\caption{Show captured pieces on start }
	\label{fig:CapturedPieces2}
\end{figure}
You can configure this window so that it is displayed at startup.

\begin{figure}[H]
	\centering
	\includegraphics[scale=1.0]{CapturedPieces3.png}
	\caption{Show all captured pieces }
	\label{fig:CapturedPieces3}
\end{figure}

Figure \ref{fig:CapturedPieces3} shows all captured pieces. Black is one queen, one knight a two pawns ahead.

\begin{figure}[H]
	\centering
	\includegraphics[scale=1.0]{CapturedPieces4.png}
	\caption{Show captured pieces as difference}
	\label{fig:CapturedPieces4}
\end{figure}
Figure \ref{fig:CapturedPieces4} shows the same information as difference.

The button \includegraphics[scale=0.5]{balance_unbalance.png} switches between both views.


\section{Electronic chessboards} \label{ElectronicChessBoard}
BearChess supports two electronic chessboards: Certabo boards and Millennium boards via the ChessLink module. Both boards communicates via an USB port as Serial COM port or via Bluetooth. Which COM port is used is not the same for all computers and can change over time, especially if you have more than one COM port available on your computer.

\subsection{Configure Certabo boards} \label{ConfigureCertabo}
\begin{figure}[H]
	\centering
	\includegraphics[scale=1.0]{Certabo1.png}
	\caption{Open a window to configure Certabo boards }
	\label{fig:Certabo1}
\end{figure}

Certabo requires two configuration steps. The used COM port and a calibration to detect the chess pieces. If your PC supports Bluetooth and you own the bluetooth module from Certabo, you can also connect to the Certabo board via Bluetooth. If you want to do this, check the option "Bluetooth" first, before opening the configuration windows. 

\begin{figure}[H]
	\centering
	\includegraphics[scale=1.0]{Certabo6.png}
	\caption{Search for Bluetooth }
	\label{fig:Certabo6}
\end{figure}

Read chapter \textbf{\ref{CertaboBluetooth}  \nameref{CertaboBluetooth}} on page \pageref{CertaboBluetooth} for more information.

\begin{figure}[H]
	\centering
	\includegraphics[scale=1.0]{Certabo2.png}
	\caption{Configure Certabo boards }
	\label{fig:Certabo2}
\end{figure}

Select the COM port (BT for Bluetooth) if you know them or let the <auto> selection. 
Click on \includegraphics[scale=0.5]{connect.png} to verify your selection.

\begin{figure}[H]
	\centering
	\includegraphics[scale=0.9]{Calibrate1.png}
	\caption{COM port successful detected}
	\label{fig:Calibrate1}
\end{figure}

If you select an invalid COM port, you will receive the following error message:

\begin{figure}[H]
	\centering
	\includegraphics[scale=0.9]{Calibrate2.png}
	\caption{Invalid COM port }
	\label{fig:Calibrate2}
\end{figure}

If you select <auto> and no board is found, you will receive the following error message:

\begin{figure}[H]
	\centering
	\includegraphics[scale=0.9]{Calibrate3.png}
	\caption{No board found }
	\label{fig:Calibrate3}
\end{figure}

\textbf{Hint:} You can always use the selection <auto>, but if you have more than one COM port available, it can always take some seconds until the right one is recognized.

\subsubsection{Configure Bluetooth} \label{CertaboBluetooth}

If the "Bluetooth" option is set, the following window will be displayed for a short time if you open the configuration dialog.

\begin{figure}[H]
	\centering
	\includegraphics[scale=0.8]{MillenniumChessLink10.png}
	\caption{Search for Bluetooth}
	\label{fig:CertaboBT10}
\end{figure}

During this time or if you connect to the first time, Windows may ask you to connect to the module. In this case, confirm this. 

\subsubsection{Calibration}
At the first start, BearChess needs a calibration to identify your chessmen. A new calibration is only required if you use another set of chessmen.\\
Click on \includegraphics[scale=0.08]{chessboard_base.png} to open the calibration dialog.
\begin{figure}[H]
	\centering
	\includegraphics[scale=1.0]{CalibrateBase.png}
	\caption{Calibrate base position }
	\label{fig:CalibrateBase}
\end{figure}
If all pieces on the right position click the accept button. When a calibration is running, you will see the chessboard LEDs flashing each row. Please wait until all LEDs are off and the confirm dialog appears.

\begin{figure}[H]
	\centering
	\includegraphics[scale=1.0]{Calibrate4.png}
	\caption{Calibration finished }
	\label{fig:Calibrate4}
\end{figure}

\textbf{Hint:} If the calibration never seems to end, check that the chessmen are correctly placed in the middle of the squares.

\subsubsection{Connect}
\begin{figure}[H]
	\centering
	\includegraphics[scale=1.0]{Certabo3.png}
	\caption{Connect to Certabo}
	\label{fig:Certabo3}
\end{figure}
When the configuration is complete, you can connect to your chessboard.
In the lower right corner a new button appears, which allows you to easily connect or disconnect your board. The current status is also written.

\begin{figure}[H]
	\centering
	\includegraphics[scale=0.8]{Certabo4.png}
	\caption{Connected}
	\label{fig:Certabo4}
\end{figure}

\begin{figure}[H]
	\centering
	\includegraphics[scale=0.8]{Certabo5.png}
	\caption{Disconnected}
	\label{fig:Certabo5}
\end{figure}

\subsection{Configure Millennium ChessLink} \label{ConfigureChessLink}
\begin{figure}[H]
	\centering
	\includegraphics[scale=1.0]{MillenniumChessLink1.png}
	\caption{Open a window to configure Millennium boards }
	\label{fig:MillenniumChessLink1}
\end{figure}

For your Millennium ChessLink you may need to configure the COM port used. You can also change how the LEDs should light up.\\
If your PC supports Bluetooth, you can also connect to the ChessLink module via Bluetooth. If you want to do this, check the option "Bluetooth" first, before opening the configuration windows. 
\begin{figure}[H]
	\centering
	\includegraphics[scale=1.0]{MillenniumChessLink9.png}
	\caption{Search for Bluetooth }
	\label{fig:MillenniumChessLink9}
\end{figure}
Read chapter \textbf{\ref{Bluetooth}  \nameref{Bluetooth}} on page \pageref{Bluetooth} for more information.

\begin{figure}[H]
	\centering
	\includegraphics[scale=0.9]{MillenniumChessLink2.png}
	\caption{Configure Millennium ChessLink }
	\label{fig:MillenniumChessLink2}
\end{figure}

Select the COM port if you know them or let the <auto> selection.
Click on \includegraphics[scale=0.5]{connect.png} to verify your selection.

\begin{figure}[H]
	\centering
	\includegraphics[scale=0.9]{MillenniumChessLink3.png}
	\caption{COM port successful detected }
	\label{fig:MillenniumChessLink3}
\end{figure}

If you select an invalid COM port, you will receive the following error message:

\begin{figure}[H]
	\centering
	\includegraphics[scale=0.9]{MillenniumChessLink4.png}
	\caption{Invalid COM port }
	\label{fig:MillenniumChessLink4}
\end{figure}

If you select <auto> and no board is found, you will receive the following error message:

\begin{figure}[H]
	\centering
	\includegraphics[scale=0.9]{MillenniumChessLink5.png}
	\caption{No board found }
	\label{fig:MillenniumChessLink5}
\end{figure}

\textbf{Hint:} You can always use the selection <auto>, but if you have more than one COM port available, it can always take some seconds until the right one is recognized.\\

You can change how the LEDs should light up. The brightness and whether the LEDs should flash alternately or synchronously when indicating moves.

\subsubsection{Configure Bluetooth} \label{Bluetooth}

If the "Bluetooth" option is set, the following window will be displayed for a short time if you open the configuration dialog.

\begin{figure}[H]
	\centering
	\includegraphics[scale=0.8]{MillenniumChessLink10.png}
	\caption{Search for Bluetooth}
	\label{fig:MillenniumChessLink10}
\end{figure}

During this time or if you connect to the first time, Windows may ask you to connect to the Millennium ChessLink module. In this case, confirm this.

\subsubsection{Connect}
\begin{figure}[H]
	\centering
	\includegraphics[scale=1.0]{MillenniumChessLink6.png}
	\caption{Connect to Millennium ChessLink}
	\label{fig:MillenniumChessLink6}
\end{figure}
When the configuration is complete, you can connect to your chessboard.
In the lower right corner a new button appears, which allows you to easily connect or disconnect your board. The current status is also written.

\begin{figure}[H]
	\centering
	\includegraphics[scale=0.8]{MillenniumChessLink7.png}
	\caption{Connected}
	\label{fig:MillenniumChessLink7}
\end{figure}

\begin{figure}[H]
	\centering
	\includegraphics[scale=0.8]{MillenniumChessLink8.png}
	\caption{Disconnected}
	\label{fig:MillenniumChessLink8}
\end{figure}


\section{Play a game}

\begin{figure}[H]
	\centering
	\includegraphics[scale=1.0]{NewGame1.png}
	\caption{Opens a window to play a new game}
	\label{fig:NewGame1}
\end{figure}

If you start a new game you can select the opponents and the time control.

\subsection{Select opponent} \label{SelectOpponent}

\begin{figure}[H]
	\centering
	\includegraphics[scale=0.8]{NewGame2.png}
	\caption{Play a new game}
	\label{fig:NewGame2}
\end{figure}

Select the opponents for white \includegraphics[scale=0.4]{KingW.png} and black \includegraphics[scale=0.4]{KingB.png}. "Player" means a human opponent. You can select "Player" for white and black if you want to play a game against another human opponent on your chessboard. You can select an engine for white and black if you want to play a pure engine match.\\
\includegraphics[scale=0.4]{user_silhouette.png} is a shortcut to select "Player".\\


\textbf{Hint:} You cannot use an electronic chessboard for a pure engine match.\\

\includegraphics[scale=0.4]{cog.png} opens the engine configuration dialog. This is the same dialog as for load and manage engines. However, all changes are only used for this game and are not saved permanently.\\
Below the engine selection, there are up to three information which gives you a quick overview of the current engine configuration.

\begin{itemize}
	\item \textbf{Ponder} if the engine supports pondering, the icons \includegraphics[scale=0.4]{tick.png} and \includegraphics[scale=0.4]{delete.png} shows the current state.
	\item \textbf{Elo} if the engine supports allows to restrict the Elo performance, the current value is shown.
	\item  \includegraphics[scale=0.4]{book_open.png} for "yes" and \includegraphics[scale=0.4]{book.png} for "no" shows the current state of using an opening book.
\end{itemize}

\subsection{Playing in relaxed mode}

\includegraphics[scale=0.8]{BearChessIcon.png} If you play against one engine (not available if two engines play against each other) you can check 'Relaxed'.  Read more in chapter \textbf{\ref{RelaxedMode}  \nameref{RelaxedMode}} on page \pageref{RelaxedMode}.


\subsection{Start from}
Select your start position. The default is the base position, but you can start from any position on the board.


\subsection{Time control}

\begin{figure}[H]
	\centering
	\includegraphics[scale=1.0]{TimeControl.png}
	\caption{Time control}
	\label{fig:TimeControl}
\end{figure}

\begin{itemize}
	\item \textbf{Time per game} limits the entire game to the given minutes.
	\item \textbf{Time per game with increment} limits the entire game to the given minutes but give extra seconds for every move.
	\item \textbf{Time per given move} limits the game for moves in a certain time frame, e.g. 40 moves in 60 minutes.
	\item \textbf{Average time per move} requires the engine to execute the moves in an average time of seconds or minutes.
	\item \textbf{Adapted time} gives the program the same average time in which the player makes his moves.	
\end{itemize}


\subsection{Tournament mode}
If you play in tournament mode, no auxiliary information (score, best line) is displayed by the engine.

\subsection{Allow to take a move back}
It is worth activating this option for a training session or when playing against a strong engine. If you are using an electronic chessboard, just take back the last move. The LEDs show you the next previous move. Follow them until you make the right move.\\
If you are not using an electronic chessboard, you can use the move controls below the board.

\begin{figure}[H]
	\centering
	\includegraphics[scale=1.0]{MoveControl.png}
	\caption{Move controls}
	\label{fig:MoveControl}
\end{figure}


\subsection{Extra time for human player}
Add extra minutes for human opponents.

\subsection{Start the clock after the move is executed on the electronic board}
Avoids a time gap to execute the moves of the engines on the board.

\subsection{Configuration for startup game} \label{startupgame}

For your convenience, you can configure and save a game definition that will be used if you have the "Run startup game at startup" option turned on.  Read more in chapter \textbf{\ref{runstartupgame}  \nameref{runstartupgame}} on page \pageref{runstartupgame}.

\begin{itemize}
	  \item \includegraphics[scale=0.5]{layer_save.png} Saves the definition.
  	  \item \includegraphics[scale=0.5]{layer_open.png} Loads the definition.
\end{itemize}

\section{Move list}
The move list shows all moves of your game.
\begin{figure}[H]
	\centering
	\includegraphics[scale=1.0]{movelist1.png}
	\caption{Simple move list}
	\label{fig:moveList1}
\end{figure}

\includegraphics[scale=0.5]{text_smallcaps.png} changes the font size in three different levels. Small font for all, larger font only for the played moves, larger font for all.\\

When you play against an engine BearChess collects more information.\\

\includegraphics[scale=0.5]{show_detail.png} expands the move list and shows the value and best move calculated by the engine.
\begin{figure}[H]
	\centering
	\includegraphics[scale=1.0]{movelist2.png}
	\caption{Move list}
	\label{fig:moveList2}
\end{figure}

\includegraphics[scale=0.5]{script_add.png} expands the move list and shows the value and the full best move list calculated by the engine.

\begin{figure}[H]
	\centering
	\includegraphics[scale=0.7]{movelist3.png}
	\caption{Full move list}
	\label{fig:moveList3}
\end{figure}

\section{Easy start or restart of a game} \label{easyStart}

The button \includegraphics[scale=0.2]{bearchess_2.png} is an easy way to play a game. If your not active playing a game, the following dialog appears:

\begin{figure}[H]
	\centering
	\includegraphics[scale=1.0]{resetToBasePosition2.png}
	\caption{Start a game}
	\label{fig:resetToBasePosition2}
\end{figure}

\begin{itemize}
	\item \textbf{Start a new game} opens the new game dialog (see \textbf{\ref{SelectOpponent}  \nameref{SelectOpponent}} on page \pageref{SelectOpponent})
	\item \textbf{Continue} enables the continuation of a game, e.g. just loaded from the database (see \textbf{\ref{ContinueAGame}  \nameref{ContinueAGame}} on page \pageref{ContinueAGame}.)
	\item \textbf{Cancel} the dialog.
\end{itemize}

If your playing a game, the following dialog appears:

\begin{figure}[H]
	\centering
	\includegraphics[scale=0.9]{resetToBasePosition.png}
	\caption{Reset to base position}
	\label{fig:resetToBasePosition}
\end{figure}

\begin{itemize}
	\item \includegraphics[scale=0.2]{bearchess.png} opens the new game dialog (see \textbf{\ref{SelectOpponent}  \nameref{SelectOpponent}} on page \pageref{SelectOpponent})
	\item \includegraphics[scale=0.1]{chessboard_base.png} restarts the game with the current opponents and time control.
    \item \includegraphics[scale=0.5]{control_stop.png} stops the current game.
    \item \includegraphics[scale=0.5]{control_play_blue.png} continues
    \item \includegraphics[scale=0.5]{diskette.png} stops the current game and opens the save game dialog
\end{itemize}

\subsubsection{Continue a game} \label{ContinueAGame}
If you want to play a game and continue on another day, you can do the following:
\begin{itemize}
    \item Save the game into the database and close BearChess.
    \item Open BearChess and load the game from the database on another day.
    \item Press the button \includegraphics[scale=0.2]{bearchess_2.png} and choose "Continue" \includegraphics[scale=0.5]{control_play_blue.png}.
\end{itemize}
The game will continue with the same settings of opponent and timing control, setting the clock to the same time as you saved the game.

\subsubsection*{Electronic chessboard}
If you use an electronic chessboard, connect to the board \textbf{before} you load the game from the database. When you click on \includegraphics[scale=0.5]{control_play_blue.png}  "Continue"
the following window appears.
\begin{figure}[H]
	\centering
	\includegraphics[scale=0.9]{waitPosition.png}
	\caption{Waiting}
	\label{fig:waitPosition}
\end{figure}
Now, place all chessmen to the right position. BearChess will recognize it and continues automatically.

\subsection{Influence of BearChess in engine games}
Chess engines generally do not have the ability to give up or to recognize a draw, e.g. by repeating moves. In a match between two engines BearChess checks the moves and the current rating. 
If a draw is detected, e.g. by a move repetition or insufficient material, the game is ended.

\begin{figure}[H]
	\centering
	\includegraphics[scale=1.0]{Games3.png}
	\caption{Draw by repetition}
	\label{fig:Games3}
\end{figure}

\begin{figure}[H]
	\centering
	\includegraphics[scale=1.0]{Games5.png}
	\caption{Draw by insufficient material}
	\label{fig:Games5}
\end{figure}

If both programs give at least a -4 or +4 in their scores over several moves, the game is finished.

\begin{figure}[H]
	\centering
	\includegraphics[scale=1.0]{Games4.png}
	\caption{Won by score}
	\label{fig:Games4}
\end{figure}


\section{Relaxed mode} \label{RelaxedMode}
This is a special mode that supports you when you play against a strong engine and the UCI strength limitation option is not suitable for you or is not offered by the engine. You can play in relaxed mode against every engine except the MessChess computer simulation. \\
It is available only if you are playing against an engine and a electronic chessboard is connected.\\
When you activate it, some options are fixed and cannot be changed. Time control is 10 seconds average time per move, undoing a move is allowed and tournament mode is turned off.

\begin{figure}[H]
	\centering
	\includegraphics[scale=1.0]{relaxed1.png}
	\caption{Time control}
	\label{fig:Relaxed1}
\end{figure}

Now, when you starts the game and the next engine move will bring you in a bad position, BearChess takes a look to find another move. In this case the BearChess icon occurs in the engine window and the bestline information is hidden.

\begin{figure}[H]
	\centering
	\includegraphics[scale=0.8]{relaxed2.png}
	\caption{BearChess in action}
	\label{fig:Relaxed2}
\end{figure}
 It takes some seconds more to evaluate an alternate move for the engine. BearChess will not make any other move if the current move captures one of your pieces or all alternative moves 
 are not realistic.\\ \\
Some technical things: The control of the opponent's engine and the evaluation of alternative moves is performed by the engine "Teddy". You will see a Teddy.exe on the Windows Task Manager.
"Teddy" is not a clone of any other chess engines and does not simply call the multivaration modus 
of the opponent engine. That's why you can use almost any engine without multivariation support.\\\\
{\color{red}*} The algorithm for finding alternative moves is not perfect and will be continuously improved in the next versions.

\section{Analysis mode} \label{AnalyzeMode}


\begin{figure}[H]
	\centering
	\includegraphics[scale=1.0]{AnalyzeMode.png}
	\caption{Analysis mode}
	\label{fig:AnalyzeMode}
\end{figure}


One outstanding feature compared to other GUIs is the analysis mode. The idea behind it is how a player analyses his games or interesting positions. The figures are quickly rearranged or moves are made that do not always conform to the rules.\\
For example, the player does not want to go back the last three or four moves to try out a new variant. He sets up the new position directly on the board.\\
Or you want to analyze a played game, but according to the chess rules. You can simply try some variations and BearChess will take you back to the last played position from the game.\\
Or you are more of a beginner and want to practice a typical endgame, e.g. king and pawn against king. You then want to know quickly if this move leads to a win or if the opponent can draw.\\
You can achieve all this with the analysis mode.\\


{\color{red}*} It may happen that not all engines are suitable for analysis. Problems have occurred so far with Arasan 20.3 (not with Arasan 22.2) or with Komodo (the reason with Komodo is still under investigation). You cannot use MessChess computer emulation engines.

\subsection{Free analyse} \label{AnalyzeMode1}

\subsection{With electronic chessboard}

You get the biggest advantage if you have connected an electronic chessboard with a piece recognition. Connect your electronic chessboard before you start the analysis mode.\\ 
When you start the analysis mode, you will be asked to select a supporting analysis engine.

\begin{figure}[H]
	\centering
	\includegraphics[scale=0.9]{AnalyzeMode2.png}
	\caption{Selection for an analysis engine}
	\label{fig:AnalyzeMode2}
\end{figure}

The engine immediately starts to analyze the current position. If you want, you can add more engines.\\
Now you can remove or add figures or make moves on your chessboard as you like. The engine window shows immediately the current analysis. The current color results from which figure was moved last. To change the color, simply lift a figure of the current color and put it back in its place. The analysis will then start for the other color. Especially in endgames it can be important to know which color is on the move. Think of the king and pawn versus king endgame.

\begin{figure}[H]
	\centering
	\includegraphics[scale=1.0]{AnalyzeMode3.png}
	\caption{Stop analysis mode}
	\label{fig:AnalyzeMode3}
\end{figure}

\subsection{Without electronic chessboard}
You can also use the analysis mode without an electronic chessboard. If you do not want to start from the base position, you should first set up the desired position via Setup position.\\
When you start the analysis mode, you will be asked to select a supporting analysis engine.

\begin{figure}[H]
	\centering
	\includegraphics[scale=0.9]{AnalyzeMode2.png}
	\caption{Selection for an analysis engine}
	\label{fig:AnalyzeMode2_2}
\end{figure}

The engine immediately starts to analyze the current position. If you want, you can add more engines.\\
You can rearrange individual figures by clicking on the figure and then on the target field.\\
If you press the right mouse button on a field, a context menu appears.

\begin{figure}[H]
	\centering
	\includegraphics[scale=1.0]{AnalyzeMode4.png}
	\caption{Analysis context menu}
	\label{fig:AnalyzeMode4}
\end{figure}

Click on one the figure to place in on the selected field. If the is a piece on the field, the button \includegraphics[scale=0.3]{toggle.png} removes it from the board. With the two clocks symbol can you switch the current color.

\begin{figure}[H]
	\centering
	\includegraphics[scale=1.0]{AnalyzeMode3.png}
	\caption{Stop analysis mode}
	\label{fig:AnalyzeMode3_2}
\end{figure}

\subsection{Analyse a game} \label{AnalyzeMode2}

This mode is available only if you are connected to an electronic chessboard and a game has been loaded from the games manager. In this case BearChess will ask you if you want to analyze a game.\\
\begin{figure}[H]
	\centering
	\includegraphics[scale=1.0]{AnalyzeGame1.png}
	\caption{Ask for analyze a game}
	\label{fig:AnalyzeGame1}
\end{figure}

When you confirm, you will be prompt to place all pieces on the base position. Do this and press Ok.
\begin{figure}[H]
	\centering
	\includegraphics[scale=1.0]{AnalyzeGame2.png}
	\caption{Wait for base position}
	\label{fig:AnalyzeGame2}
\end{figure}
As next, you will be asked to select a supporting analysis engine.

\begin{figure}[H]
	\centering
	\includegraphics[scale=0.9]{AnalyzeMode2.png}
	\caption{Selection for an analysis engine}
	\label{fig:AnalyzeMode2_3}
\end{figure}

Make your moves as indicated to the move list. The engine immediately starts to analyze the current position. The current move on the move list is highlighted  and the board shows the position. In this example ,the engine prefers the move bishop c1 d2 instead pawn g2 g3.

\begin{figure}[H]
	\centering
	\includegraphics[scale=0.7]{AnalyzeMode5.png}
	\caption{Analyze after the 4th move}
	\label{fig:AnalyzeMode5}
\end{figure}

\begin{figure}[H]
	\centering
	\includegraphics[scale=0.7]{AnalyzeMode6.png}
	\caption{Board after the 4th move}
	\label{fig:AnalyzeMode6}
\end{figure}

The electronic chessboard indicates the move proposal from the engine.\\
If you make a valid move outside the move list on your electronic chessboard, e.g. if you follow the suggestions of the engine, the engine continues the analysis but you will not see the move on the chessboard on the screen. You can make all valid moves back and forth on your electronic chessboard but the chessboard on the screen and the move list show you the last position in the game. If you go back to a valid position from the game the chessboard and move list will continue.


\section{Engine duel} \label{EngineDuel}

\begin{figure}[H]
	\centering
	\includegraphics[scale=1.0]{EngineDuel1.png}
	\caption{Engine duel}
	\label{fig:EngineDuel}
\end{figure}

\begin{itemize}
	\item \includegraphics[scale=0.5]{6-2-chess-png.png} Start a new engine duel.
	\item \includegraphics[scale=0.5]{file_manager.png} Manage or continue an engine duel. 
\end{itemize}

\subsection{Start a new engine duel}

\begin{figure}[H]
	\centering
	\includegraphics[scale=0.7]{EngineDuel2.png}
	\caption{New engine duel}
	\label{fig:EngineDuel2}
\end{figure}

In a duel, two engines play one or more games against each other. You can define the number of games and whether to change colors after each game. First, give the event an useful name and select the engines. 
With \includegraphics[scale=0.5]{file_manager.png} you can change the database where the games are stored.

\subsubsection{Engine duel information}
When you have started the duel, the following information window appears:

\begin{figure}[H]
	\centering
	\includegraphics[scale=1.0]{EngineDuel3.png}
	\caption{Engine duel information}
	\label{fig:EngineDuel3}
\end{figure}

\includegraphics[scale=0.5]{control_stop_blue.png} ask you if you want to stop the current duel.

\begin{figure}[H]
	\centering
	\includegraphics[scale=1.0]{EngineDuel4.png}
	\caption{Stop engine duel}
	\label{fig:EngineDuel4}
\end{figure}

If you confirm, the current duel and game is stored and you can continue the duel at the same position.

\begin{figure}[H]
	\centering
	\includegraphics[scale=1.0]{EngineDuel5.png}
	\caption{Engine duel finished}
	\label{fig:EngineDuel5}
\end{figure}

The final result is shown after the duel is finished.

\subsection{Manage engine duels}

\begin{figure}[H]
	\centering
	\includegraphics[scale=0.6]{EngineDuel6.png}
	\caption{Manage engine duels}
	\label{fig:EngineDuel6}
\end{figure}

Shows an overview of all duels. The top list shows duel information and below it all played games of the duel. The example shows one finished duel and one running duel where the first game is not finished.

\begin{itemize}
	\item \includegraphics[scale=0.5]{control_play_blue.png} Continue a running duel with the last unfinished game at the same position.
	\item \includegraphics[scale=0.5]{6-2-chess-png.png} Load the selected duel configuration as template for a new duel.
	\item \includegraphics[scale=0.5]{control_repeat_blue.png} Deletes all games of the selected duel and runs the duel again.
    \item \includegraphics[scale=0.5]{bin.png} Deletes the selected duel and all associated games.
        \item \includegraphics[scale=0.5]{database_delete.png} Deletes all duels and all associated games.
\end{itemize}

\section{Engine tournament} \label{EngineTournamentl}

\begin{figure}[H]
	\centering
	\includegraphics[scale=1.0]{EngineTournament.png}
	\caption{Engine tournament}
	\label{fig:EngineTournament}
\end{figure}

\begin{itemize}
	\item \includegraphics[scale=0.5]{cup_gold.png} Start a new engine tournament.
	\item \includegraphics[scale=0.5]{file_manager.png} Manage or continue an engine tournament. 
\end{itemize}

\subsection{Start a new engine tournament}

\begin{figure}[H]
	\centering
	\includegraphics[scale=0.4]{EngineTournament1.png}
	\caption{New engine tournament}
	\label{fig:EngineTournament1}
\end{figure}

The left side configures the tournament where on the right side you can select the participants.

\begin{figure}[H]
	\centering
	\includegraphics[scale=0.8]{EngineTournament2.png}
	\caption{Participants}
	\label{fig:EngineTournament2}
\end{figure}
The number of participants influences the number of games. The example shows four participants.

\begin{itemize}
	\item \includegraphics[scale=0.5]{control_back.png} Add a new participant
	\item \includegraphics[scale=0.5]{control_play.png} Remove a participant. 
	\item \includegraphics[scale=0.5]{control_rewind.png} Add all engines as participant. 
	\item \includegraphics[scale=0.5]{control_fastforward.png} Remove all participants. 
    \item \includegraphics[scale=0.5]{cog.png} Configure selected participant. 
\end{itemize}

\begin{figure}[H]
	\centering
	\fbox{\includegraphics[scale=0.9]{EngineTournament3.png}}
	\caption{Configuration}
	\label{fig:EngineTournament3}
\end{figure}

Currently, two types of tournaments are supported.
\begin{itemize}
	\item \textbf{RoundRobin} Everyone plays against everyone
	\item \textbf{Gauntlet} One (deliquent) plays against everyone
\end{itemize}

\textbf{Cycles} determines the number of passes. The example shows four participants for a roundrobin tournament. Six games are played if everyone plays against everyone. With two cycles there are twelve games.\\

Select the deliquent for a gauntlet tournament.  The example shows four participants for a gauntlet tournament. Three games are played if the deliquent plays against everyone. With two cycles there are six games


\begin{figure}[H]
	\centering 
	\fbox{\includegraphics[scale=0.8]{EngineTournament4.png}}
	\caption{Gauntlet}
	\label{fig:EngineTournament4}
\end{figure}


With \includegraphics[scale=0.5]{file_manager.png} you can change the database where the games are stored.

\subsubsection{Engine tournament information}

When you have started the tournament, the following information window appears:

\begin{figure}[H]
	\centering
	\includegraphics[scale=1.0]{EngineTournament5.png}
	\caption{Engine tournament information roundrobin}
	\label{fig:EngineTournameent5}
\end{figure}

\begin{figure}[H]
	\centering
	\includegraphics[scale=1.0]{EngineTournament6.png}
	\caption{Engine tournament information gauntlet}
	\label{fig:EngineTournameent6}
\end{figure}

\includegraphics[scale=0.5]{control_stop_blue.png} ask you if you want to stop the current tournament.

\subsection{Manage engine tournaments}

\begin{figure}[H]
	\centering
	\includegraphics[scale=0.6]{EngineTournament7.png}
	\caption{Manage engine tournaments}
	\label{fig:EngineTournament7}
\end{figure}

Shows an overview of all tournaments. The top list shows tournament information and below it all played games of the tournament. The example shows one finished tournament and one running tournament where the second game is not finished.

\begin{itemize}
	\item \includegraphics[scale=0.5]{control_play_blue.png} Continue a running tournament with the last unfinished game at the same position.
	\item \includegraphics[scale=0.5]{cup_gold.png} Load the selected tournament configuration as template for a new tournament.
	\item \includegraphics[scale=0.5]{control_repeat_blue.png} Deletes all games of the selected tournament and runs the tournament again.
	\item \includegraphics[scale=0.5]{bin.png} Deletes the selected tournament and all associated games.
	\item \includegraphics[scale=0.5]{database_delete.png} Deletes all tournaments and all associated games.
\end{itemize}

\section{Setup position} \label{SetupPosition}

\begin{figure}[H]
	\centering
	\includegraphics[scale=1.0]{SetupPosition1.png}
	\caption{Opens a window for position setup}
	\label{fig:SetupPosition1}
\end{figure}

Setup a new position is very easy. You can do it with or without the support of an electronic chessboard.

\subsection{With support of an electronic chessboard}

\textbf{First connect} to your electronic chessboard before you run the setup. This is important, because the further behaviour of the program depends on it.
The small board starts with the current position.

\begin{figure}[H]
	\centering
	\includegraphics[scale=0.8]{SetupPosition4.png}
	\caption{Setup a new position with an electronic chessboard}
	\label{fig:SetupPosition4}
\end{figure}

Now you can also give the electronic chessboard an arbitrary position, which is immediately displayed on the small chessboard.\\
Don't forget to set the castling rights and current color.

\begin{figure}[H]
	\centering
	\includegraphics[scale=1.0]{castling.png}
	\caption{Castling rights and current color}
	\label{fig:castling2}
\end{figure}

With the button \includegraphics[scale=0.5]{accept_button.png} the new position is placed on the chessboard.\\
\textbf{{\color{red}*}} It is not completely checked whether the position is valid. This also applies to castling rights.

\subsection{Without support of an electronic chessboard}

\textbf{First} ensure you are \textbf{not connect} to your electronic chessboard before you run the setup. This is important, because the further behaviour of the program depends on it.\\
The small board starts with the current position.

\begin{figure}[H]
	\centering
	\includegraphics[scale=0.55]{SetupPosition2.png}
	\caption{Setup a new position without an electronic chessboard}
	\label{fig:SetupPosition2}
\end{figure}

The input box shows the fen position. You can insert a new position in the input field and click on "Set" to place it on the small board.

\begin{figure}[H]
	\centering
	\includegraphics[scale=0.55]{SetupPosition3.png}
	\caption{FEN input box}
	\label{fig:SetupPosition3}
\end{figure}

To place a piece on the board, select the corresponding icon. The desired color is not important here.\\
\includegraphics[scale=1]{WhiteP.png} \includegraphics[scale=1]{WhiteN.png} 
\includegraphics[scale=1]{WhiteB.png} \includegraphics[scale=1]{WhiteK.png}
\includegraphics[scale=1]{WhiteQ.png} \includegraphics[scale=1]{WhiteR.png}\\

With a left click on the small board you place a white piece, with a right click a black piece.
If you click on field with a piece on it, you remove it.\\
Don't forget to set the castling rights and current color.

\begin{figure}[H]
	\centering
	\includegraphics[scale=1.0]{castling.png}
	\caption{Castling rights and current color}
	\label{fig:castling}
\end{figure}

There are three buttons to quickly set the base position or empty the board.
\begin{itemize}
	\item \includegraphics[scale=1]{Board64black.png} Remove all pieces from the board.
	\item \includegraphics[scale=0.3]{Array.png} Put all pieces on their base position.
	\item \includegraphics[scale=0.5]{Undo.png} Reset to start position.
\end{itemize}

With the button \includegraphics[scale=0.5]{accept_button.png} the new position is placed on the chessboard.\\
\textbf{{\color{red}*}} It is not completely checked whether the position is valid. This also applies to castling rights.

\section{Run startup game on start} \label{runstartupgame}

If you start BearChess and want to start a game immediately, chapter \textbf{\ref{startupgame}  \nameref{startupgame}} on page \pageref{startupgame} describes how to configure and save the definition.

\begin{figure}[H]
	\centering
	\includegraphics[scale=1.0]{runonstartup.png}
	\caption{Run a game on startup}
	\label{fig:RunOnStartup1}
\end{figure}

This is like turning on a chess computer that is immediately ready to play.



\section{Games} \label{games}

\begin{figure}[H]
	\centering
	\includegraphics[scale=1.0]{Games1.png}
	\caption{Games}
	\label{fig:Games1}
\end{figure}

\subsection{Save}

BearChess stores all games in a database file. When you save a game for the first time, you must first select a database file.\\
The save dialog is prefilled with the known data. But you can correct them before saving.

\begin{figure}[H]
	\centering
	\includegraphics[scale=1.0]{Games2.png}
	\caption{Save a game}
	\label{fig:Games2}
\end{figure}

{\color{red}*} Currently BearChess does not support comments or variants in the PGN notation.

\subsection{Show and load}

BearChess stores all games in one PGN file. The current name is shown in the title bar.

\begin{figure}[H]
	\centering
	\includegraphics[scale=0.6]{Games6.png}
	\caption{Games window}
	\label{fig:Games6}
\end{figure}

If you double click on a row, the game is displayed on the board.

\begin{itemize}
	\item \includegraphics[scale=0.5]{database_add.png} Create a new games database.
	\item \includegraphics[scale=0.5]{folder_database.png} Open an existing games database.
	\item \includegraphics[scale=0.5]{bin.png} Delete selected game. You cannot delete a game witch is a participant of a duel or tournament. Use the duel or tournament manager instead.
	\item \includegraphics[scale=0.5]{saved_imports.png} Import games from a PGN file {\color{red}*}.	
	\item \includegraphics[scale=0.5]{clipboard_sign_out.png} Copy selected game into clipboard.		
	\item \includegraphics[scale=0.5]{link.png} Filter games that correspond to the current board position.	
	\item \includegraphics[scale=0.5]{table_filter.png} Opens a window for further filter options.		
	\item \includegraphics[scale=0.5]{database_delete.png} Delete all games from the database, including duel and tournament information.
	\item \includegraphics[scale=0.5]{door_out.png} Close the window.
\end{itemize}

{\color{red}*} The import is very slow. It will be improved in the next version.

{\color{red}**} Currently BearChess does not support a search function, only filter.

\section{Engine window}

Each loaded engine is listed in the engine window.

\begin{figure}[H]
	\centering
	\includegraphics[scale=0.8]{EngineWindow1.png}
	\caption{Two loaded engines}
	\label{fig:EngineWindow1}
\end{figure}

If the engine allows to configure its ELO number, it appears under the name.
\begin{itemize}
	\item \includegraphics[scale=0.5]{control_pause_blue.png} \includegraphics[scale=0.5]{control_play_blue.png} Pause the engine or continue.
	\item \includegraphics[scale=0.5]{toggle_expand.png} Add a info line.
	\item \includegraphics[scale=0.5]{toggle.png} Remove a info line.
	\item \includegraphics[scale=0.5]{control_power_blue.png} Close the engine. Not visible if you play a game.
	\item \includegraphics[scale=0.5]{cog.png} Opens the configuration dialog.	
	\item \includegraphics[scale=0.5]{eye.png} Hide engine information.		
\end{itemize}

If you have configured that currently the best move should be displayed, the analysis of the topmost engine is taken.

\begin{figure}[H]
	\centering
	\includegraphics[scale=0.7]{EngineWindow2.png}
	\caption{Two loaded engines}
	\label{fig:EngineWindow2}
\end{figure}

\section{Extended engine support}  \label{ExtendedSupport}

Another outstanding feature compared to other GUIs is the extended engine support. You can load one or more engines at any time to assist you in a game against another engine or another player.
\begin{enumerate}
	\item Start a new game against an engine.
	\item Load a second engine for assistent.
\end{enumerate}

The following figure gives you an example. You play a game against Stockfish and Komodo gives hints for the next best move. 

\begin{figure}[H]
	\centering
	\includegraphics[scale=0.7]{Extended1.png}
	\caption{Play against Stockfish with support from Komodo}
	\label{fig:Extended1}
\end{figure}

Even if you are not playing a game and are on mode "Easy playing", just load some engines and make your moves.

\section{Important To Know}

\subsection{Install a new version}
All your configration settings for BearChess is not stored in the directory from which you start BearChess. For an installation of a new version of BearChess you can delete the complete BearChess directory and unpack the new version there. Or they unpack the new version in a separate directory and still have the same configuration.

\subsection{MessChess engines}
These engines simluate a chess computer the configuration is limited to the capabilities of them. Not everything that is possible with normal engines can be done here. 

\subsection{Certabo: Calibration}
For the first time, the engine assumes that all chessmen are on their initial position and the extra queens on d3 (white queen) and d6 (black queen).

\subsection{Certabo: Pawn conversion to a queen}
If you have performed the calibration without extra queens and are performing a pawn conversion with a extra queen on your board for the first time, the program needs a few seconds to identify the new piece. Please wait until the LEDs are off. The new piece code is stored. There is no delay next time.

\section{Trouble shooting}

\subsection{The chess moves are not or not correctly displayed}
\begin{itemize}
	\item Check the correct COM port in the configuration dialog.
	\item Reconnect to the chessboard.	
	\item Check the position of the chessmen. Moves are only accepted if the chessmen are on the correct square. Fields with missing or wrong figure light up.
\end{itemize}


\section{Known Issues}
\begin{itemize}
    \item Relaxed mode and the game analysis are new features. There are certainly still some errors in it.
    \item Synchronization between move list and chessboard works only in the simple move list.
    \item \textbf{Setup position} It is not completely checked whether the position is valid. This also applies to castling rights.
	\item Some windows may overlap for the first time.
	\item If you want to play an engine match with the same engine for black and white you have to install the engine twice with a different name.
\end{itemize}

\section{Next Steps}

\begin{itemize}
		\item Extend the game analysis.
		\item Improve the COM port handling.
  	    \item Improve the performance of the opening book.
	\end{itemize} 

\pagebreak

\section{Changelog}

\subsection{Version 0.5.5.2 =\textgreater 0.5.5.3}
\begin{itemize}
	\item Fix communication with Millennium ChessLink
\end{itemize}

\subsection{Version 0.5.5.1 =\textgreater 0.5.5.2}
\begin{itemize}
	\item Fix closing loaded engines.
    \item Fix handling coach engines.
\end{itemize}

\subsection{Version 0.5.5.0 =\textgreater 0.5.5.1}
\begin{itemize}
	\item Fix level configuration for MessChess chess computer emulation by Franz Huber.	
\end{itemize}

\subsection{Version 0.5.0.0 =\textgreater 0.5.5.0}
\begin{itemize}
	\item New analyze mode for played games.
	\item Relaxed mode against any UCI engine.
	\item Support of Certabo Avatar UCI engines.
	\item Support of MessChess chess computer emulation by Franz Huber.	
	\item Improve the response time to changes on the electronic chessboard.
	\item Improvement of figure recognition at Certabo.			
	\item Optional parameter and logo for UCI engines.	
	\item Hide engine output.		
	\item Minor fixes.
\end{itemize}

\subsection{Version 0.4.4.0 =\textgreater 0.5.0.0}
\begin{itemize}
	\item Engine duels and tournaments.
	\item Filter games.
	\item Minor fixes.	
\end{itemize}

\subsection{Version 0.4.3.1 =\textgreater 0.4.4.0}
\begin{itemize}
	\item Changes and fixes in the move list.
    \item Now, "Captured pieces" counts if you capture a promoted figure.
	\item Changed the font for moves to a monospaced font.
	\item Fixed a bug where you could remove a king in analysis mode without an electronic chessboard.
	\item Minor fixes.	
\end{itemize}

\subsection{Version 0.4.3.0 =\textgreater 0.4.3.1}
\begin{itemize}
	\item Fixed a bug that occurred in the analysis mode without an electronic chessboard.
\end{itemize}

\subsection{Version 0.4.2.3 =\textgreater 0.4.3.0}
\begin{itemize}
	\item Give "Player" a first and last name
    \item Show best moves on electronic chessboards in analyse mode.
	\item Improvement of the COM port detection.
    \item Improvement in automatic window arrangement.
\end{itemize}

\subsection{Version 0.4.2.2 =\textgreater 0.4.2.3}
\begin{itemize}
	\item Fixed a bug with the display of the taken back move on the chessboard.
\end{itemize}

\subsection{Version 0.4.2.1 =\textgreater 0.4.2.2}
\begin{itemize}
	\item Fixed a bug running additional engines in a game.	
\end{itemize}

\subsection{Version 0.4.2.0 =\textgreater 0.4.2.1}
\begin{itemize}
	\item Fixed bug in number conversion that crashes the program.
\end{itemize}

\subsection{Version 0.4.1.0 =\textgreater 0.4.2.0}
\begin{itemize}
	\item Bluetooth support for Certabo chessboards.
	\item Playing games in tournament mode.
	\item Minor fixes.	
\end{itemize}

\subsection{Version 0.4.0.0 =\textgreater 0.4.1.0}
\begin{itemize}
	\item Using a database file instead PGN.
	\item Allows you to continue a previously saved game.
	\item Extended information in the move list window.
    \item Extended information in the engines window.
	\item Minor fixes.	
\end{itemize}

\subsection{Version 0.3.3.0 =\textgreater 0.4.0.0}
\begin{itemize}
	\item New time configuration "Adapted time".
	\item Implementation of the "Start the clock after the move is executed on the electronic board" time setting.
    \item Bluetooth support for Millennium ChessLink.
	\item Recognizing the base position as a new start of a game.
	\item Configuration of a startup game.
	\item Changing the behaviour of "Easy playing" mode
	\item Correction and extension of the evaluating of UCI configuration values.
	\item Correction at the start of a new game (Engine was not started).
	\item Correction if the engine starts with white.
	\item Minor fixes.	
\end{itemize}

\subsection{Version 0.3.2.0 =\textgreater 0.3.3.0}
\begin{itemize}
    \item Support of opening books for engines.
    \item Opening book: When castling, the correct squares are displayed.
	\item Improvements and bug fixes in "New Game" dialog.
	\item Minor fixes.	
\end{itemize}

\subsection{Version 0.3.1.1 =\textgreater 0.3.2.0}
\begin{itemize}
	\item Show check and mate signs in move list.
	\item Show captured pieces.	
    \item Save configuration for "Black pieces in front".
	\item Improvements in the configuration chessmen.
	\item Improvements to install a new engine.
\end{itemize}

\subsection{Version 0.3.1.0 =\textgreater 0.3.1.1}
\begin{itemize}
	\item \textbf{Hotfix} Error at pure engine match.
	\item Improvements in the handling of UCI configuration values.
\end{itemize}

\subsection{Version 0.3.0.0 =\textgreater 0.3.1.0}
\begin{itemize}
	    \item Improvements in the calibration for Certabo chessboards.
	    \item Fixed an error on pawn conversation.
   	    \item Fixed an error on position setup via electronic chessboard.
\end{itemize}

\end{document}